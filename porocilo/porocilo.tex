\documentclass[11pt,a4paper]{article}

\usepackage[slovene]{babel}
\usepackage[utf8x]{inputenc}
\usepackage{graphicx}
\usepackage{pdfpages}
\usepackage{hyperref}

\pagestyle{plain}

\begin{document}
\title{Poročilo pri predmetu \\
Analiza podatkov s programom R}
\author{Valentina Gnamuš}
\maketitle

\section{Izbira teme}
Za temo projekta sem si izbrala enakopravnost med spoloma. Moj namen je analizirati domnevne razlike med ženskim in moškim delom prebivalstva na področjih dela, denarja, izobrazbe, zdravja, prostega časa in moči (politične, gospodarske). Pri tem se bom osredotočila predvsem na evropske države, ker pa je tema aktualna tudi drugod po svetu, bo za primerjavo prikazano še stanje v kakšni neevropski državi.

\flushleft{Pri zbiranju podatkov mi bodo v pomoč naslednje spletne strani:}

\begin{itemize}
\item{\url{ http://data.worldbank.org/topic/gender#boxes-box-topic_cust_sec}}
\item{\url{http://www.stat.si/letopis/LetopisVsebina.aspx?poglavje=12&lang=si&leto=2012}}
\item{\url{http://epp.eurostat.ec.europa.eu/portal/page/portal/statistics/search_database}}
\item{\url{http://w3.unece.org/pxweb}}
\item{\url{http://eige.europa.eu}}
\end{itemize}
\section{Obdelava, uvoz in čiščenje podatkov}

V tej fazi projekta sem uvozila osem podatkovnih tabel, ki sem jih pridobila na nekaterih izmed zgoraj navedenih virov. Natančneje: veliko večino podatkov (vse tabele oblike CSV) sem črpala s spletne strani Evropskega inštituta za spolno enakopravnost, ostali podatki pa so s spletne strani Evropskega statističnega urada.

Zbrane tabele sem uredila, odstranila odvečne znake in zanje napisala program za uvoz. 
Ker je za datoteke oblike HTML čiščenje podatkov nekoliko zahtevnejše kot za CSV, sem za zadnji dve tabeli v mapi lib napisala pomožen program, ki je podatke očistil. Ta program sem nato klicala v uvozi.r, kjer sem napisala tudi ostale funkcije za uvoz podatkov.

Podatke sleherne uvožene tabele sem prikazala tudi z grafi. Navadno sem za prikaz izbrala najpomembnejšo komponento tabele (na primer iz tabele \verb-index_znanje- sem izbrala podatke za komponento "Knowledge", ki zajema vse ostale (pod)komponente). 

Za risanje grafov sem napisala svoj program Grafi.r, ki sem ga shranila v mapo slike.
Za prikaz podatkov iz prvih šest tabel (oblike CSV) sem izbrala stolpični graf, saj je tako najbolj razvidno, kakšna je stopnja enakopravnosti (glede na ustrezno področje) v posameznh  državah. Ta je merjena z indeksom enakopravnosti, posebne merske enote, s pomočjo katere na enostaven način lahko razberemo (ne)enakosti med spoloma (indeks=100 pomeni popolno enakopravnost, indeks=0 pa neenakopravnost).

Podatke iz zadnjih dveh tabel sem združila v enem grafu, saj sem želela primerjati število opravljenih delovnih ur moških in žensk v Sloveniji. Ker me je zanimala samo raven razlike v posameznem letu, sem si za grafično sliko izbrala točkovni graf.

Grafe sem v zaključku z ukazom pdf(...) prevedla v pdf obliko.

\includepdf[pages={1-7}]{../slike/grafi.pdf}

\section{Analiza in vizualizacija podatkov}

Podatke iz tabel, ki sem jih v prejšnji fazi uvozila, sem tokrat prikazala še na zemljevidu.
Ker podatki za Slovenijo niso bili podani po regijah, sem lahko prikazala samo podatke za celo Evropo. To sem naredila tako, da sem iz zemljevida sveta izluščila vse evropske države. Moji podatki so bili izmerjeni za članice Evropske unije v letu 2010, tako da za nekatere evropske države ne poznam rezultatov domnevnih analiz. Te države so zato na zemljevidih obarvane belo.\\
\vspace{5mm}
Prvi zemljevid kaže enakopravnost spolov glede na preživljanje prostega časa po posameznih državah. Kot lahko vidimo, je večja neenakopravnost v vzhodnem delu EU, ko se pomikamo proti severu in zahodu, pa se enakopravnost veča.

\includegraphics[width=\textwidth]{../slike/zemljevidcas.pdf}

\newpage
Drugi zemljevid prikazuje neenakopravnost spolov glede na delo. To pomeni, da so v indeks enakopravnosti vključeni vsi dejavniki, ki vplivajo na neenakopravnost in izključeni tisti, ki niso ključni za ugotavljanje problema. Torej primerjamo posameznike obeh spolov glede na
življenjsko delovno dobo, zdravje in varnost pri delu, izobraževanje na delovnem mestu itd.

Neenakopravnost pri tej spremenljivki ni tako velika kot pri prostem času. Kot najmanj izenačena izstopa Bolgarija s 37\% enakopravnostjo. Sicer so tipično slabše spet države na vzhodu in jugu EU, proti severu in vzhodu se situacija boljša.

\includegraphics[width=\textwidth]{../slike/zemljeviddelo.pdf}

\newpage

Tretji zemljevid prikazuje enakopravnost glede na denar, torej glede na dohodek in druge finančne vire. Spet je situacija jasna. Državljani živeči v vzhodnih, severovzhodnih in jugovzhodnih državah EU oz na Iberskem polotoku so zopet manj enakopravni od tistih, ki živijo v srednji, zahodni oziroma severni Evropi. Romunija ima tukaj najslabšo situacijo, sledijo ji Bolgarija, Latvija in Litva. Tudi Poljska, Slovaška in Estonija se niso najbolje odrezale: njihova stopnja enakopravnosti glede na denarne prihodke je manj kot 50\%.

\includegraphics[width=\textwidth]{../slike/zemljeviddenar.pdf}

\newpage

Glede na moč so si po spolu prebivalci držav EU najbolj različni. Indeks enakopravnosti glede na moč za vseh 27 držav EU je le 38\%. Švedska, Finska, Danska, Nizozemska in Francija so najbolj enakopravne. V negativno smer izstopata predvsem Ciper in Luksemburg z 12,2 in 14,7 procentnim deležem enakopravnosti.

\includegraphics[width=\textwidth]{../slike/zemljevidmoc.pdf}

\newpage

Peti zemljevid nam pokaže enakopravnost med spoloma glede na zdravje (dostopnost zdravljenja, zdrava leta življenja vsakega posameznika, ...). Vidno je, da so prebivalci prav vseh držav po spolu zelo enakopravni, kar zadeva zdravje, saj ima najslabša država več kot 75 odstotno stopnjo enakopravnosti. Kakorkoli, najmanj enakopravne države so še vedno tiste, ki gradijo vzhodni del meje EU - torej vse od vključno Estonije pa do vključno Bolgarije -- ter Portugalska, od tega Latvija po neenakopravnosti še dodatno izstopa. Najbolj enakopravno zdravi pa so Belgijci, Nizozemci, Britanci in Irci.

\includegraphics[width=\textwidth]{../slike/zemljevidzdravje.pdf}

\newpage

Šesti in zadnji zemljevid prikazuje enakopravnost, merjeno glede na znanje oziroma izobrazbo ljudi. Danska, Velika Britanija, Švedska, Finska in Nizozemska imajo po spolu in znanju najbolj enakopravno prebivalstvo.Nasprotno pa so Romunija, Bolgarija, Italija in Portugalska tu najšibkejše.

\includegraphics[width=\textwidth]{../slike/zemljevidznanje.pdf}

Slovenija pri vseh področjih merjenja enakopravnosti zaseda zlato sredino. Nikjer ne izstopa, vendar je ponekod celo nad povprečjem. Kot država je torej na zelo dobri poti do spolno enakopravnega prebivalstva.


%\includegraphics{../slike/povprecna_druzina.pdf}

\section{Napredna analiza podatkov}
V zadnji fazi sem se odločila, da bom pogledala, ali imajo BDP, stopnja zaposlenosti v posamezni državi in rodnost kaj skupnega s spolno enakopravnostjo. Želela sem ugotoviti tudi, katere države so si med sabo podobne po enakopravnem prebivalstvu.



%\includegraphics{../slike/naselja.pdf}

\end{document}
